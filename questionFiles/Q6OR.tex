%%%%%%%%%%%%%%%%%% The OR question
	
\begin{parts}
%%%%%%%%%%%%%%%%%%%%%%%%%%%%%%%%%%%%%%%%%%%%%%%%%%%%%%%%%%%%%%%%%%%%%%		
\part[4]Simplify the Boolean function $F(w, x, y, z) = \sum(0, 1, 2, 4, 5, 6, 8,9, 12, 13, 14)$. 


\begin{solution}
	
\end{solution}	
%%%%%%%%%%%%%%%%%%%%%%%%%%%%%%%%%%%%%%%%%%%%%%%%%%%%%%%%%%%%%%%%%%%%%%


% \sisetup{per-mode = symbol} % this will produce unit like : mV/ps
\part[4] A transmission line has the following per-unit length parameters: \(L = \SI{0.5}{\micro\henry\per\meter}\), \(C = \SI{200}{\pico\farad\per\meter}\), \(R =  \SI{4}{\ohm\per\meter} \) and \( G = \SI{0.02}{\siemens\per\meter} \). Calculate the propagation constant and characteristics impedance of this line. If the line is \SI{30}{\centi\meter} long, what is the attenuation in \SI{}{\deci\bel}.

\begin{solution}
	
\end{solution}		
%%%%%%%%%%%%%%%%%%%%%%%%%%%%%%%%%%%%%%%%%%%%%%%%%%%%%%%%%%%%%%%%%%%%%%		
\part[2\half] If you only want the unit as symbol mode like  \(R = {\sisetup{per-mode = symbol} \SI{4}{\ohm\per\meter} }\) instead of the conventional unit  \(R =  \SI{4}{\ohm\per\meter} \), wrap the unit in braces: as shown below.

\begin{verbatim}
{
\sisetup{per-mode = symbol}
\si{\milli\volt\per\pico\second}
}
\end{verbatim}


\begin{solution}
	
\end{solution}		
%%%%%%%%%%%%%%%%%%%%%%%%%%%%%%%%%%%%%%%%%%%%%%%%%%%%%%%%%%%%%%%%%%%%%%
\end{parts}

